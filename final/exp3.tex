\begin{center}
\textbf{实验三:数据库应用系统的开发}
\end{center}

\section{实验目的}
初步掌握数据库应用系统分析设计的基本方法;进一步提高分析与解决问题
的综合能力;初步掌握数据库建模工具的使用方法;熟悉掌握 CS 或 BS 结构的
数据库应用系统开发的整个过程。

\section{实验预备内容}
\begin{enumerate}
  \item 阅读教材《数据库系统概论》设计与应用开发篇。
  \item 阅读实验使用的数据库管理系统的相关帮助文档。
  \item 学习数据库建模工具的使用;
  \item 学习某种应用编程开发工具及其连接访问数据库的方法;
  \item 学习设计、开发数据库应用系统的相关知识.
\end{enumerate}

\section{实验环境}
\begin{itemize}
  \item OS: Linux;
  \item DBMS: OpenGauss;
  \item Programming language: Python, JavaScript;
  \item Others: Vim
\end{itemize}

\section{实验内容}

背景自定义,使用某种应用开发编程工具(Java、Python、PHP、VC 等)结
合华为云平台数据库 openGauss 完成一个数据库应用系统(BS、CS 模式均可)
的分析、设计与实现。

基本要求:能实现对数据库中数据的插入、删除、修改、查询、统计查询等
功能,做到界面友好、使用方便。掌握程序访问数据库中数据的技术方法,进一
步提高分析与解决问题的综合能力。

实验三如果系统工作量较大可 2-3 人团队合作完成,但要选定组长负责并
明确任务分工,保障每人的工作量饱满。

\section{系统选题需求情况及任务分工情况说明}
(团队合作需细化说明每个队员负责的具体功能模块,工作量占比)

\section{系统的概念数据模型设计}
(E-R图)

\section{系统中每张表的说明}

\section{系统运行环境配置}
(安装操作说明,前端与后台数据库连接用到的关键语句说明)

\section{系统主要功能界面介绍}
(需使用截图)

\section{实验结果总结}
(分析系统运行效果,说明系统优缺点及改进方向)

\section{编程工作总结}
(系统开发所付出的努力、面临的困难,自学了哪些相关知识;
自己负责什么模块,遇到什么问题怎么解决的,有没有自己创新的设计,
开发的体会与收获等。请认真完成,不少于 500 字。
注意:分组完成的同学只说明自己完成的工作,不要写别人的工作)

说明:团队合作的同学,组长的报告需写完整,组员报告实验内容一至五项可以标
注为``见组长报告'',组员和组长都需在报告封面上组长的姓名后括号标注组长。
